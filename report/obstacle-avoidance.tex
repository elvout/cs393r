\documentclass[11pt]{article}
\usepackage{amsmath}
\usepackage{enumitem}
\usepackage{fullpage}
\usepackage{geometry}
\usepackage{hyperref}
\usepackage[utf8]{inputenc}

\geometry{top=1in, bottom=1in, left=1in, right=1in}

\DeclareMathOperator{\sgn}{sgn}

\title{CS 393R Autonomous Robots \\ \large Assignment 1: Obstacle Avoidance}
\author{Elvin Yang, Aidan Dunlap}
\date{September 20, 2021}

\begin{document}
\maketitle

\section{Questions}

Let
\begin{itemize}
    \item
    $l$, $w$, $b$, $d$, and $m$ represent the car's length, width, wheel base,
    track width, and obstacle safety margin, respectively.

    \item
    $r$ represent the car's radius of turning. Negative $r$ signifies a right
    turn.

    \item
    $p = (x, y)$ represent an observed point in the base\_link reference frame.

    \item
    $a = \frac{l - b}{2}$ represent the distance between the front axle and the
    front end of the car as well as the distance between the rear axle and the
    rear end of the car.

    % \item
    % $c_1, c_2, c_3, c_4$ represent the corners of the car in each respective
    % quadrant of the base\_link reference frame.
\end{itemize}

\begin{enumerate}[leftmargin=*]
    \item
    Which point of the car traces an arc of maximum radius? What is that radius,
    in terms of $r$ and other car parameters?

    The front corner of the car that is farthest from the center of turning
    traces an arc of maximum radius. It has radius
    $$ r_{max} = \sgn(r) \sqrt{\left(l + a\right)^2 + \left(\lvert r \rvert + \frac{w}{2}\right)^2} $$

    \item
    Which point of the car traces an arc of minimum radius? What is that radius,
    in terms of $r$ and other car parameters?

    The rear corner of the car closest to the center of turning traces an
    arc of minimum radius. It has radius
    $$ r_{min} = \sgn(r) \sqrt{\left(l - a \right)^2 + \left(\lvert r \rvert - \frac{w}{2}\right)^2} $$

    \item
    Under what conditions will $p$ first hit the inner side (the side closer to
    the center of turning) of the car as the car drives forward?

    % mention case where r == \inf?

    Let $r_1, r_2$ represent the distances from the endpoints of the inner side
    of the car to the center of turning, where $r_1 < r_2$.  Let $d_p$ represent
    the distance from $p$ to the center of turning.

    If $r_1 < d_p < r_2$, then $p$ will hit the inner side of the car.

    \item
    Under what conditions will $p$ first hit the front of the car as the car
    drives forward?

    % mention case where r == \inf?

    Let $r_1, r_2$ represent the distances from the endpoints of the front side
    of the car to the center of turning, where $r_1 < r_2$.  Let $d_p$ represent
    the distance from $p$ to the center of turning.

    If $r_1 < d_p < r_2$, then $p$ will hit the front side of the car.

    \item
    Under what conditions will $p$ first hit the outer side (the side farther to
    the center of turning) of the car as the car drives forward?

    The instantaneous direction of a point on the car is perpendicular to the
    vector from the center of turning to that point. Without loss of generality,
    for left turns, the instantaneous velocity of the rear corner of the outer
    side has a negative $y$ component and the instantaneous velocity of the
    front corner of the outer side has a positive $y$ component.

    If $p$ is sufficiently close to the rear corner of the outer side of the
    car, it will hit the outer side of the car.

    % mathematical explanation as well?
    % y in -a to a
    % x max is r_3 - |r|

    \item
    What is the maximum distance (the free path length) the car can move forward
    along the arc before it hits the point $p$?

    Let $\theta$ represent the angle formed by the base link, center of turning,
    and $p$.
    % how to find the hit location?
    Let $\beta$ represent the angle formed by the base link, center of turning,
    and location on the perimeter of the car where $p$ will hit.  Define
    $\alpha$ as $\theta - \beta$, or the angle subtended by the base link when
    the car hits $p$.

    The free path length is $\alpha \lvert r \rvert$, the arc length of the
    angle subtended by the base link.

    \item
    If the current velocity of the car is $v$, and the maximum magnitude of
    deceleration is $a$, what is the minimum stopping distance of the car?
    \begin{align*}
        \lVert v_f \rVert^2 &= \lVert v_i \rVert^2 - 2ad_{stop} \\
        0 &= \lVert v \rVert^2 - 2 a d_{stop} \\
        d_{stop} &= \frac{\lVert v \rVert^2}{2a}
    \end{align*}
\end{enumerate}

\section{Algorithm Details}

\subsection{Parameters}

We define the following as constants:
\begin{itemize}
    \item Size of the car.
    \\ We measured these with a tape measure. We used these for a variety of uses from collision detection to path planning, multiplied by a margin-of-safety coefficient (c > 1) that increased the size of the car.
    \item Safety margin around the car.
    \\ We manually searched for a safety margin  coefficient that we thought managed both safety and flexibility effectively.  
    \item Limits of the car's steering angle.
    \\ We measured this with a tape measure with the car's maximum controllable turning radius. This was used for a lower/upper bound on the car's turning angle which we used to search for potential paths. 
    \item Steering angle step size for testing different constant-curvature arcs.
    \\ This was a hand-tuned parameter that we adjusted to balance turning flexibility and route calculation latency
    \item The maximum magnitude of velocity, acceleration, and deceleration.
    \\ We used this to plan our time-optimal control paths. We never used or tested the maximum velocity of the car and instead stuck to safe but also challenging velocities.  
    \item Actuation latency.
    \\ We used this for latency compensation (see more below)
    \item Location of the LIDAR sensor.
    \\ We used this to translate LIDAR points back into the base link/predicted base link reference frame
\end{itemize}

Our path-selection algorithm uses the following properties of each path option:
\begin{itemize}
    \item The free path length.
    \\The distance the car could travel such that it both does not collide with an obstacle and also doesn't drive past the closest point on the circle to the target. 
    \item The point on the free path closest to the navigation goal.
    \\We used this to determine the smallest absolute displacement between the car on its path and the target.
\end{itemize}

\subsubsection{Parameter Tuning}

We measured the physical dimensions of the car and the location of the LIDAR
sensor in the base link reference frame. We tuned actuation latency by trial and
error, testing multiple latencies until the car was able to travel in a straight
path of 2 meters within an acceptable margin of error.

\subsection{Limitations and Improvements}

We observed that our latency compensation was able to achieve a relatively low
displacement error when we only take into account actuation latency. However, we
noticed that the navigation target would shift a lot in the visualization. We
are not sure whether this is due to observation latency or noise and precision
errors in our transformation code, and we plan on correcting for observation
latency to rule it out as a factor.

\bigskip
\noindent
We observed that our path-finding algorithm worked well without factoring in
clearance to the path-selection heuristic since the safety margin enforces an
lower bound on the clearance for each path. However, we plan on merging
clearance code into our main development branch to make our algorithm more
robust in complex environments.

% code performance?

\subsection{Challenges}

Upon running the car for the first time, we found that the steering was heavily
biased to the left. We had to set the
\texttt{steering\_angle\_to\_servo\_offset} higher than the suggested range for
steering to be symmetric.

We encountered a lot of bugs due to infinite radii in the straight-line-path
case. Most of our functions perform separate computations for this edge case.

\section{Repository Link}
\href{http://github.com/elvout/cs393r-a1}{http://github.com/elvout/cs393r-a1}

\section{Demonstration}
Videos \href{https://drive.google.com/drive/folders/1Z8qGX_YRjMb96KEpW027cGvmGeLOB8BU}{here}

\section{Contributions}

When we began, Aidan worked on obstacle detection and path selection and Elvin
worked on time optimal control and latency compensation. We spent a significant
amount of time working as a group to figure out math, integrate our
components together, and test our code on the physical car.

\end{document}
