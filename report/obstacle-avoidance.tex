\documentclass[11pt]{article}
\usepackage{amsmath}
\usepackage{enumitem}
\usepackage{fullpage}
\usepackage{geometry}
\usepackage[utf8]{inputenc}

\geometry{top=1in, bottom=1in, left=1in, right=1in}

\DeclareMathOperator{\sgn}{sgn}

\title{CS 393R Autonomous Robots \\ \large Assignment 1: Obstacle Avoidance}
\author{Aidan Dunlap, Elvin Yang}
\date{September 20, 2021}

\begin{document}
\maketitle

\section{Questions}

Let
\begin{itemize}
    \item
    $l$, $w$, $b$, $d$, and $m$ represent the car's length, width, wheel base,
    track width, and obstacle safety margin, respectively.

    \item
    $r$ represent the car's radius of turning. Negative $r$ signifies a right
    turn.

    \item
    $p = (x, y)$ represent an observed point in the base\_link reference frame.

    \item
    $a = \frac{l - b}{2}$ represent the distance between the front axle and the
    front end of the car as well as the distance between the rear axle and the
    rear end of the car.

    % \item
    % $c_1, c_2, c_3, c_4$ represent the corners of the car in each respective
    % quadrant of the base\_link reference frame.
\end{itemize}

\begin{enumerate}[leftmargin=*]
    \item
    Which point of the car traces an arc of maximum radius? What is that radius,
    in terms of $r$ and other car parameters?

    The front corner of the car that is farthest from the center of turning
    traces an arc of maximum radius. It has radius
    $$ r_{max} = \sgn(r) \sqrt{\left(l + a\right)^2 + \left(\lvert r \rvert + \frac{w}{2}\right)^2} $$

    \item
    Which point of the car traces an arc of minimum radius? What is that radius,
    in terms of $r$ and other car parameters?

    The rear corner of the car closest to the center of turning traces an
    arc of minimum radius. It has radius
    $$ r_{min} = \sgn(r) \sqrt{\left(l - a \right)^2 + \left(\lvert r \rvert - \frac{w}{2}\right)^2} $$

    \item
    Under what conditions will $p$ first hit the inner side (the side closer to
    the center of turning) of the car as the car drives forward?

    % mention case where r == \inf?

    Let $r_1, r_2$ represent the distances from the endpoints of the inner side
    of the car to the center of turning, where $r_1 < r_2$.  Let $d_p$ represent
    the distance from $p$ to the center of turning.

    If $r_1 < d_p < r_2$, then $p$ will hit the inner side of the car.

    \item
    Under what conditions will $p$ first hit the front of the car as the car
    drives forward?

    % mention case where r == \inf?

    Let $r_1, r_2$ represent the distances from the endpoints of the front side
    of the car to the center of turning, where $r_1 < r_2$.  Let $d_p$ represent
    the distance from $p$ to the center of turning.

    If $r_1 < d_p < r_2$, then $p$ will hit the front side of the car.

    \item
    Under what conditions will $p$ first hit the outer side (the side farther to
    the center of turning) of the car as the car drives forward?

    The instantaneous direction of a point on the car is perpendicular to the
    vector from the center of turning to that point. Without loss of generality,
    for left turns, the instantaneous velocity of the rear corner of the outer
    side has a negative $y$ component and the instantaneous velocity of the
    front corner of the outer side has a positive $y$ component.

    If $p$ is sufficiently close to the rear corner of the outer side of the
    car, it will hit the outer side of the car.

    % mathematical explanation as well?
    % y in -a to a
    % x max is r_3 - |r|

    \item
    What is the maximum distance (the free path length) the car can move forward
    along the arc before it hits the point $p$?

    Let $\theta$ represent the angle formed by the base link, center of turning,
    and $p$.
    % how to find the hit location?
    Let $\beta$ represent the angle formed by the base link, center of turning,
    and location on the perimeter of the car where $p$ will hit.  Define
    $\alpha$ as $\theta - \beta$, or the angle subtended by the base link when
    the car hits $p$.

    The free path length is $\alpha \lvert r \rvert$, the arc length of the
    angle subtended by the base link.

    \item
    If the current velocity of the car is $v$, and the maximum magnitude of
    deceleration is $a$, what is the minimum stopping distance of the car?
    \begin{align*}
        \lVert v_f \rVert^2 &= \lVert v_i \rVert^2 - 2ad_{stop} \\
        0 &= \lVert v \rVert^2 - 2 a d_{stop} \\
        d_{stop} &= \frac{\lVert v \rVert^2}{2a}
    \end{align*}
\end{enumerate}
\end{document}
